\documentclass{article}
\usepackage{hyperref}
\hypersetup{
    colorlinks=true,
    linkcolor=blue,
    filecolor=magenta,      
    urlcolor=cyan,
    pdftitle={Overleaf Example},
    pdfpagemode=FullScreen,
    }

\title{Running it all on Tomcat : \\ Geonetwork \& Geoserver}
\author{Advaith C A}

\begin{document}
    \maketitle
    \section*{Introduction}
        \hspace{1.5em}This article is written with the intention of it acting as a way to reproduce the results I achieved in getting Geonetwork and Geoserver running on my PC (A Windows device) using their web archive (\textit{WAR}) files on a tomcat server. I will provide a brief on the steps involved and the versions of the involved software, and the hardware I have used to get it all up and running.
    \section*{Hardware}
        \hspace{1.5em}The hardware specifications of my PC:
        \begin{itemize}
            \item AMD Ryzen 7 5800H with Radeon Graphics 3.20 GHz
            \item A dedicated RTX 3060 GPU
            \item 16 GB RAM
            \item 1 TB SSD (with roughly 260 GB Free space)
            \item Windows 11 v23H2
        \end{itemize}
    \section*{Versions of softwares used}
    \begin{itemize}
        \item Java 8
        \item Tomcat 9.0.89 (Supports running GeoNetwork 4.2.6)
        \item GeoNetwork 4.2.6 (Stable version)
        \item Geoserver 2.22.5 (Supports Java 8 SDK)
        \item Elasticsearch 7.17.15 (Supported by GeoNetwork 4.2.6)
    \end{itemize}
        \hspace{1.5em}The above versions were chosen after consulting the \href{https://docs.geonetwork-opensource.org/4.2/install-guide/}{documentation} of GeoNetwork.

    \section*{Steps}
        \subsection*{Downloading the files and their placements}
            \begin{enumerate}
                \item Download \href{https://www.oracle.com/java/technologies/javase/javase8u211-later-archive-downloads.html}{Java 8} and install it. Make sure that \textit{JAVA\_HOME} has been added to the Environment as a variable pointing to the installed JDK.
                \item Download the \href{https://dlcdn.apache.org/tomcat/tomcat-9/v9.0.89/bin/apache-tomcat-9.0.89.zip}{zip file} for Tomcat 9.0.89. Extract it to some location in your PC.
                \item Download the \href{https://sourceforge.net/projects/geonetwork/files/GeoNetwork_opensource/v4.2.6/geonetwork.war/download}{GeoNetwork.war} file for version 4.2.6. Add this file to the \emph{webapps} folder in the extracted Apache folder.
                \item Download the ElasticSearch \href{https://artifacts.elastic.co/downloads/elasticsearch/elasticsearch-7.17.15-windows-x86_64.zip}{zip file}. Extract it somewhere in your PC, preferrably somewhere near your previously extracted Apache folder to make navigation easier.
                \item Download GeoServer 2.22.5 \href{https://sourceforge.net/projects/geoserver/files/GeoServer/2.22.5/geoserver-2.22.5-war.zip}{WAR file}. It will be a zip file, extract and add the WAR file to webapps folder in the Apache directory.    
            \end{enumerate}
        \subsection*{Running the server}
            \begin{enumerate}
                \item Open a CMD or Terminal window in the \emph{bin} directory of Apache directory.
                \item Run the command \verb|startup.bat|
                \item Open another CMD or Terminal window in the \emph{bin} directory of the ElasticSearch directory.
                \item Run the command \verb|elasticsearch.bat|
            \end{enumerate}
        \subsection*{Troubleshooting}
            \begin{itemize}
                \item If after doing the above steps, nothing happens, please ensure that the versions are the same as that which I have used.
                \item If the search bar does not work in GeoNetwork, Edit the \emph{elasticsearch.yml} file the config directory of ElasticSearch, disable the security. (Usually the first uncommented line, make it false instead of true).
                \item If there are too many errors, just use WSL or spin up an Ubuntu virtual machine in Oracle VirtualBox.
            \end{itemize}

\end{document}